%% Copyright 2015 BarD Software s.r.o
%% This is a showcase of Metropolis theme for Beamer
%%
%% This work is licensed under a Creative Commons Attribution-ShareAlike 4.0
%% International License (https://creativecommons.org/licenses/by-sa/4.0/).
\documentclass[12pt]{beamer}
\usetheme{metropolis}
\usepackage{polyglossia}
\setdefaultlanguage{russian}

\usepackage{booktabs}
\usepackage[scale=2]{ccicons}

%% Содержание титульного листа
\title{Metropolis}
\subtitle{Современная тема для презентации}
\date{\today}
\author{Команда Папирии}
\institute{\url{papeeria.com}}

\begin{document}

\maketitle

\begin{frame}
\frametitle{Содержание}
\setbeamertemplate{section in toc}[sections numbered]
\tableofcontents[hideallsubsections]
\end{frame}

%% =================================================================
%% Раздел "Введение"
\section{Введение}

%% Как включить тему
\begin{frame}[fragile]
\frametitle{Metropolis}

\textsc{Metropolis} -- это тема для пакета Beamer с минимальным количеством визуального шума, разработанная Маттиасом Фогельгесангом.

Тема включается стандартным способом:

\begin{verbatim}
    \documentclass{beamer}
    \usetheme{metropolis}
\end{verbatim}

Для компиляции необходимо использовать XeTeX. Все необходимые шрифты семейства \texttt{FiraSans} уже включены в шаблон.
\end{frame}

%% Как делать разделы
\begin{frame}[fragile]
\frametitle{Разделы}
Разделы группируют слайды

\begin{verbatim}    \section{Элементы}\end{verbatim}

а \textsc{Metropolis} делает для них симпатичный индикатор прогресса \ldots
\end{frame}

%% =================================================================
%% Раздел "Элементы"
\section{Элементы}

%% Типографические элементы
\begin{frame}[fragile]
\frametitle{Типографика}
Если вы набираете это
\begin{verbatim}
Тема устанавливает стили для \emph{разных} 
\alert{способов} \textbf{выделения} текста.
\end{verbatim}

\begin{center}то получаете это\end{center}

Тема устанавливает стили для \emph{разных} \alert{способов}  \textbf{выделения} текста.
\end{frame}

%% Шрифты
\begin{frame}{Какие бывают шрифты}
  \begin{itemize}
    \item Обычный
    \item \textit{Наклонный}
    \item \textsc{Капитель}
    \item \textbf{Жирный}
    \item \textbf{\textit{Жирный наклонный}}
    \item \textbf{\textsc{Жирная капитель}}
    \item \texttt{Моноширинный}
    \item \texttt{\textit{Наклонного моноширинного нет. А надо?}}
    \item \texttt{\textbf{Жирный моноширинный}}
    \item \texttt{\textbf{\textit{Жирного наклонного моноширинного тоже нет}}}
  \end{itemize}
\end{frame}

%% Таблицы
\begin{frame}{Таблицы}
  \begin{table}
    \caption{Самые большие города в мире (согласно Википедии):}
    \begin{tabular}{lr}
      \toprule
      Город & Население\\
      \midrule
      Мехико & 20,116,842\\
      Шанхай & 19,210,000\\
      Пекин & 15,796,450\\
      Стамбул & 14,160,467\\
      \bottomrule
    \end{tabular}
  \end{table}
\end{frame}

%% Двухколоночные слайды и текстовые блоки
\begin{frame}{Текстовые блоки}
Предопределены три окружения для текстовых блоков  опциональным цветом фона

\begin{columns}[T,onlytextwidth]
% Левый столбец
\column{0.5\textwidth}

\begin{block}{Default}
Block content.
\end{block}

\begin{alertblock}{Alert}
Block content.
\end{alertblock}

\begin{exampleblock}{Example}
Block content.
\end{exampleblock}

% Правый столбец
\column{0.5\textwidth}
\metroset{block=fill}

\begin{block}{Default}
Block content.
\end{block}

\begin{alertblock}{Alert}
Block content.
\end{alertblock}

\begin{exampleblock}{Example}
Block content.
\end{exampleblock}
\end{columns}
\end{frame}

%% Математические формулы
\begin{frame}{Математические формулы}
  \begin{equation*}
    e = \lim_{n\to \infty} \left(1 + \frac{1}{n}\right)^n
  \end{equation*}
\end{frame}

%% =================================================================
%% Раздел "Заключение"

\section{Заключение}

\begin{frame}{Исходники и лицензия}
Исходники этого шаблона и демонстрационной презентации находятся на гитхабе

\begin{center}\url{http://github.com/bardsoftware/template-beamer-metropolis}\end{center}

Официальный репозиторий темы Metropolis:

\begin{center}\url{http://github.com/matze/mtheme}\end{center}

Тема, шаблон и демонстрационная презентация распространяются на условиях
\href{http://creativecommons.org/licenses/by-sa/4.0/}{Creative Commons
Attribution-ShareAlike 4.0 International License}.

\begin{center}\ccbysa\end{center}

\end{frame}

\plain{Вопросы?}

%% 
\begin{frame}[allowframebreaks]
\frametitle{Список литературы}
\nocite{*}
\bibliography{References}
\bibliographystyle{abbrv}

\end{frame}

\end{document}
